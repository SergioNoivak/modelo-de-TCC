
\chapter{Materiais e Métodos}
\label{chap:materiaisemetodos}
Para o desenvolvimento desse trabalho foi utilizada a linguagem de programação javascript e o método utilizado para classificação das regras dos autômatos celulares em rede foi a medida de entropia, porque A  medida  mais  usual  para  identificar aleatoriedade em  uma  seqüência  de  eventos é a entropia \cite{hevertonAutomatoRede}. A entropia de uma seqüência de $k$ eventos é definida pela equação , onde $p_i$  é a probabilidade de ocorrência do evento $i$: 
$$ S = \sum_{i=1}^{k} p_i \cdot \log_2 p_i$$
Sendo $S$ um número decimal variando entre 0 e 1 e quanto mais próximo de um $S$ for, mais caótica é a regra do autômato celular em rede. Também segundo \cite{wolframtheoryandapplications} a inspeção dos diagramas de evolução do autômato celular também são válidas, portanto foram confeccionados diagramas para analisar a dinâmica dos autômatos celulares em rede, afim de classifica-los.

Para a confecção das amostras foi usado o método de \cite{smallworld} onde cada rede de autômato recebe um valor aleatório $p$ de forma que o modelo de Watts-Strogatz possa ser usado. Nesses autômatos, conforme $p$, um número real varia entre 0 e 1, mais aleatória é a rede de autômatos celulares em rede. 


