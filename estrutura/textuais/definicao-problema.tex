% Instituto Federal de Educação, Ciência e Tecnologia Baiano - Campus Guanambi
% 
% Modelo para Trabalho de Conclusão de Curso em LaTeX
% Superior de Análise e Desenvolvimento de Sistemas
% Alterado por: Dr. Naidson Clayr Santos Ferreira
%
% ----------------------------------------------------------------------- %
% Arquivo: definicao-problema.tex
% ----------------------------------------------------------------------- %

% DEFINIÇÃO DO PROBLEMA--------------------------------------------------------

\chapter{DEFINIÇÃO DO PROBLEMA}
\label{chap:definicaoproblema}
Certas classes de autômatos celulares possuem diversas aplicações, como, por por exemplo, em \citeonline{hevertonnewcryptography}, onde as classes de autômatos celulares caóticos possuem aplicabilidade na área da criptografia,e há também estudos do uso de autômatos celulares caóticos na detecção de imagens falsas \cite{imageforgery}. \\

Tais estudos se baseiam no comportamento dinâmico dos autômatos celulares elementares, que são aqueles que possuem a configuração mais simples possível (não trivial), sendo assim, eles são unidimensionais, homogêneos, binários, e com tamanho de raio
igual a 1. Também existem diferentes formas para classificar o comportamento dinâmico observado dos
ACs elementares, dependendo do grau de refinamento desejado \cite{acsaspectosdimanicosecomputacionais}.\\

Nesse modelo de autômato celular conhecido como elementar, há 256 possíveis e essas regras podem ser classificadas de acordo com seu comportamento dinâmico. As mesmas regras,
quando aplicado em autômatos celulares em rede, pode mudar seu comportamento dinâmico quando comparado com a classificação do autômato celular padrão devido à estrutura de conexão \cite{hevertonAutomatoRede}.\\

Portanto, uma regra classificada como caótica pode assumir classificação periódica ou alguma outra classe proposta por \citeonline{wolfram1984a} quando se altera a vizinhança para um modelo de rede no autômato celular. O problema consiste então em encontrar uma classificação adequada para esses autômatos de rede, de forma que esses possam ser utilizados em aplicações que se restringem a autômatos celulares elementares.  