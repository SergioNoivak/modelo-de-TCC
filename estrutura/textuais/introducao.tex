% Instituto Federal de Educação, Ciência e Tecnologia Baiano - Campus Guanambi
% 
% Modelo para Trabalho de Conclusão de Curso em LaTeX
% Superior de Análise e Desenvolvimento de Sistemas
% Alterado por: Dr. Naidson Clayr Santos Ferreira
%
% ----------------------------------------------------------------------- %
% Arquivo: introducao.tex
% ----------------------------------------------------------------------- %

% INTRODUÇÃO-------------------------------------------------------------------

\chapter{INTRODUÇÃO}
\label{chap:introducao}
Autômatos celulares (ACs) são sistemas distribuídos espacialmente, compostos de um
grande número de componentes simples idênticos, com conectividade local. Os ACs são
exemplos de sistemas dinâmicos discretos (variáveis, tempo e espaço), de
implementação extremamente simples,  que permite a manipulação direta de seus
parâmetros para o estudo de sua dinâmica. Por isso, os ACs se tornaram importantes
ferramentas para o estudo e modelagem de sistemas complexos reais nas mais diversas
áreas\cite{wolfram1984a}.


Esses modelos vêm sendo cada vez mais estudados e utilizados diversas aplicações \cite{wolframtheoryandapplications}, Como em criptografia \cite{hevertonnewcryptography},
modelos de propagação de incêndio florestal \cite{incendiosflorestais},
reconhecimento de linguagens \cite{reconhecimentolinguagem} e processamento de imagens \cite{celularautomataimage},.
