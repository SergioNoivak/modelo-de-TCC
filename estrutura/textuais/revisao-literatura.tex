
\chapter{Revisão de Literatura}
\label{chap:revisaoliteratura}

\subsection{Autômato Celular}
Um autômato celular é um modelo discreto estudado em diferentes áreas, tais como a
teoria da computabilidade, a matemática, e a biologia. Os autômatos celulares (ACs) foram
introduzidos nos anos 50 pelo matemático John Von Neumann, levando em conta sugestões
do físico Stanislaw Ulam \cite{wolfram1984a}. Von Neumann estava interessado nas conexões
entre a biologia e a teoria dos autômatos. Nos seus estudos, predominava a idéia do fenômeno
biológico da auto-reprodução. A questão que ele apresentava era: “Que tipo de organização
lógica é suficiente para um autômato ser capaz de reproduzir a si próprio?” \cite{monotonicidadedeclividade}
\\

De forma geral, um AC é definido por seu espaço celular e por sua regra de transição.
O espaço celular é um reticulado de N células idênticas dispostas em um arranjo ddimensional,
cada uma com um padrão idêntico de conexões locais para outras células, e com
condições de contorno \cite{acsaspectosdimanicosecomputacionais}.Cada célula do reticulado assume um estado a cada
passo de tempo, dentre um conjunto finito de estados possíveis. A regra de transição faz um
mapeamento entre células vizinhas para determinar o novo estado da célula central da
vizinhança. Após a aplicação da regra de transição em todas as células do reticulado, que pode
ser de forma síncrona ou não, é contado um passo de tempo.\\

Segundo \citeonline{hevertonAutomatoRede}, normalmente, o reticulado do
AC é submetido à regra de transição por vários passos de tempo resultando no que chamamos
de evolução temporal do AC.
Os ACs mais pesquisados são binários (dois estados possíveis),
unidimensionais e evoluem de forma síncrona.





