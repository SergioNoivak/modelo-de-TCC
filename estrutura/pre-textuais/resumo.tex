% Instituto Federal de Educação, Ciência e Tecnologia Baiano - Campus Guanambi
% 
% Modelo para Trabalho de Conclusão de Curso em LaTeX
% Superior de Análise e Desenvolvimento de Sistemas
% Alterado por: Dr. Naidson Clayr Santos Ferreira
%
% ----------------------------------------------------------------------- %
% Arquivo: resumo.tex
% ----------------------------------------------------------------------- %

% RESUMO--------------------------------------------------------------------------------

\begin{resumo}[RESUMO]
\begin{SingleSpacing}

% Não altere esta seção do texto--------------------------------------------------------
\imprimirautorcitacao. \imprimirtitulo. \imprimirdata. \pageref {LastPage} f. \imprimirprojeto\ – \imprimirprograma, \imprimirinstituicao. \imprimirlocal, \imprimirdata.\\
%---------------------------------------------------------------------------------------

Este trabalho apresenta um estudo sobre a classificação quanto a dinâmica de autômatos celulares em Rede. A necessidade surge porque certas classes de autômatos celulares elementares possuem aplicações em muitas áreas, como criptografia e para oferecer maior diversidade nessas aplicações, o uso de Autômatos Celulares em Rede só é possível através de uma classificação prévia.     \\

\textbf{Palavras-chave}: Autômatos Celulares em Rede. Classificações de Autômatos Celulares.

\end{SingleSpacing}
\end{resumo}

% OBSERVAÇÕES---------------------------------------------------------------------------
% Altere o texto inserindo o Resumo do seu trabalho.
% Escolha de 3 a 5 palavras ou termos que descrevam bem o seu trabalho 
